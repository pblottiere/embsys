%------------------------------------------------------------------------------
%<*title>
%------------------------------------------------------------------------------
\bi
\item Paul Blottiere, Télécom Bretagne Système Logiciel et Réseaux
\item Thales, Actimar
\item Partage de ce que j'ai pu voir et apprendre.
\item Linux embarqué : domaine d'avenir (drone, voiture, électroménager,
      téléphone portable, ...).
\item Présentation du but : remplir leur boite à outils.
\item Bon ingénieur : faire le bon choix au bon moment en fonction des
      conditions.
\ei
%</title>

%------------------------------------------------------------------------------
%<*organisation>
%------------------------------------------------------------------------------
\bi
    \item CM1 : histoire, définitions, normes, ...
    \item CM2 : process de boot (BIOS, bootloader), filesystem, kernel
                (architecture,
          micronoyau, monolithique, ...), mémoire, espace utilisateur
    \item CM3 - CM4 - CM5 : prog système (thread, IPC, ...)
    \item CM6 : liaison série, I2C, PCI, ...
    \item CM7 : kconfig (compil kernel, busybox), cross compilation, build
                system (Buildroot, Yocto), gdbserver, QEMU, syslog
    \item CM8 : X/X11, framebuffer, lib graphique (EFL, ...)
\ei
%</organisation>

%------------------------------------------------------------------------------
%<*histoire1>
%------------------------------------------------------------------------------
L'informatique est un monde très polémique qui regroupe tout un tas de
caractériels.
\bi
\item 1964 : Multics, OS dont but est d'acceuillir plusieurs centaines
      d'utilisateurs en même temps pour qu'un seul serveur suffise pour
      toute la région de Boston. Utilisé pdt un temps par NSA, General
      Motors, ... Ceux qui bossent dessus : MIT, Bell, General Electric.
\item 1969 : Ken Thompson développe sous Multics le 1er jeu vidéo :
      Space Travel (voyage ds le système solaire : image). Puis, Bell se
      retire du projet. Ken Thompson développe la 1ère version d'un OS
      mono-utilisateur : Unics (jeu de mot contre Multics). Codé en
      assembleur...
\ei
%</histoire1>

%------------------------------------------------------------------------------
%<*histoire2>
%------------------------------------------------------------------------------
\bi
\item 1971 : Maintenance trop compliquée avec langage assembleur... Ken
      et Dennis Ritchie (mort en 2011) envisage le Fortran mais créé le
      langage B. Mais pas terrible... Ritchie travaille sur encore un autre
      langage : New B ou C! OS maintenant en C.
\item 1975 : Unix commence à être diffusé hors des labo Bell avec sa
      version 6.
\item 1977 : Berkeley Software Distribution V1 et vi par Bill Joy.
\item 1983 : BSD v2 avec TCP/IP et autre branche Unix : System V (par
      ATandT). Commence à voir les premières versions propriétaires. En
      même temps au MIT, RMS initie un mouvement de logiciel libre en
      se lançant dans la création d'un OS libre (comme Emacs créé par
      la même personne plus tôt).
\ei
%</histoire2>

%------------------------------------------------------------------------------
%<*histoire3>
%------------------------------------------------------------------------------
\bi
\item 1987 : Minix par Andrew S. Tanenbaum à des fins pédagogique. Unix
      simple et petit pour être abordable pour ses étudiants (Finlande).
\item 1990 : Manque plus qu'un kernel à GNU. Hurd (micro-noyau) est lancé!
\item 1991 : Comme Tanenbaum ne veut intégrer des contributions à Minix
      (pour qu'il reste simple), Linus (étudiant finlandais 26ans) commence
      à écrire son propre kernel respectant les normes POSIX (hobby) : Unix
      like. Un guerre entre noyau monolithique et micronoyau. Version 0.0.2
      sorti en aout (3 mois après).
\item 1992 : Linux est intégré dans GNU!
\ei
%</histoire3>

%------------------------------------------------------------------------------
%<*histoire4>
%------------------------------------------------------------------------------
\bi
\item 1995 - 2000 : Linux de plus en plus utilisé sur server
\item 2000 : Linux de plus en plus utilisé en système embarqués
\item 2008 - 2010 : Linux de plus en plus utilisé sur tablette / téléphone
\item 2015 : Ancêtre BSD (FreeBSD, OpenBSD, NetBSD, MacOS X), ancêtre
      System V (OpenSolaris, Hp/UX, AIX), Linux
\ei
%<*histoire4>

%------------------------------------------------------------------------------
%<*normes1>
%------------------------------------------------------------------------------
Compatibilité :
\bi
\item Binaire : un logiciel créé pour système d'exploitation A peut
      être exécuté sur un système d'exploitation B sans recompilation
      (contrainte d'un processeur identique)
\item Source : un logiciel créé pour système d'exploitation A peut
      être exécuté sur un système d'exploitation B après recompilation du
      code source
\ei
Pour assurer une compatibilité des logiciels entre systèmes d'exploitation,
des normes industrielles sont établie:
\bi
\item Portable Operating System Interface : standardisation des API pour
    les UNIX-like (RMS). Plusieurs versions avec extension pour le temps
    réel (POSIX.1b) et pour les threads (POSIX.1c)
\item UNIX98 : normalisation des UNIX
\item SUSV3 : fusion de Posix et UNIX98
\ei
%</normes1>

%------------------------------------------------------------------------------
%<*normes2>
%------------------------------------------------------------------------------
La norme POSIX implique d'avoir l'ensemble des options placée avant les
paramètres. En revanche, sous GNU, les options étant précédées par des -,
peuvent être placées même après des opérandes.
ls . -a versus ls .
%</normes2>

%------------------------------------------------------------------------------
%<*opensource1>
%------------------------------------------------------------------------------
Légende de l'origine du mouvement du libre : driver d'imprimante!
\bi
\item Logiciel Libre : code source ouvert et pouvant être modifié. Le fond
      est plus philosophique : liberté des utilisateurs! Free dans le sens
      "libre" et non "gratuit" (freeware gratuit mais pas libre!)
\item Logiciel Open Source : code source ouvert pouvant être modifié. Le
      fond est plus pragmatique que philosophique. Aussi pour eviter l'aspect
      liberté (mot qui ne plaît pas forcément ) une entreprise...!
\ei
%</opensource1>

%------------------------------------------------------------------------------
%<*opensource2>
%------------------------------------------------------------------------------
En 1983, RMS (MIT) initie la création d'un OS libre. En 1990, manque le
noyau et commence à créer Hurd (micro noyau)... Comme Hurd n'avance pas
assez vite, utilise le kernel Linux! Projet de nom GNU : GNU is Not Unix
(complètement recodé). En 1985, Free Software Foundation (RMS).
Organisation visant à protéger les utilisateurs contre les logiciels
privateurs (propriétaires) et promouvoir les logiciels libres. FSF
(avec RMS) élabore des licences de distribution des logiciels : GPL et
LGPL.
%</opensource2>

%------------------------------------------------------------------------------
%<*licences1>
%------------------------------------------------------------------------------
Licence libre copyleft et non copyleft. Notion de copyleft introduite par
opposition au copyright.
%</licences1>

%------------------------------------------------------------------------------
%<*licences2>
%------------------------------------------------------------------------------
Licence libre copyleft:
\bi
\item GPL : liberté d'exécuter, d'étudier, de redistribuer, de faire
      bénéficier la communauté des versions modifiées. Point important :
      pas d'édition de lien possible entre du code GPL et du code non
      conforme! Actuellement V3!
\item LGPL : GPL amoindri permet l'édition de lien avec code non
      conforme avec la GPL. Permet à des applications propriétaires
      d'utiliser des librairies fondamentales comme la GNU C Library.
      La glibc est donc sous LGPL.
\item CeCILL, ...
\ei
Licence libre non copyleft:
\bi
\item BSD : les versions modifiées d'un logiciel libre ne sont elles
      mêmes pas nécessairement libre!
\ei
%</licences2>

%------------------------------------------------------------------------------
%<*licences3>
%------------------------------------------------------------------------------
Voir dans /usr/share/doc. Exemple avec GLIBC et NMAP. Debian suit un
\"Contrat Social\" (DFSG) envers la FSF. Cette distribution fournies
trois composants apportant chacun des paquets supplémentaires:
\bi
\item main : paquet conforme au DFSG. Autonome (sans dépendance vers des
             paquets des autres composants). Seul ces paquets sont
             considérés comme faisant partie de la distribution Debian.
\item contrib : paquet confirme au DFSG mais qui ont des dépendances en
      dehors du main
\item non-free : logiciels non conforme au DFSG.
\ei
/etc/apt/source.list
%</licences3>

%------------------------------------------------------------------------------
%<*licences4>
%------------------------------------------------------------------------------
VRMS (Virtual RMS) : informe sur la proportion de paquets libres ou non
d'une Debian. Exemple de sortie!
%</licences4>

%------------------------------------------------------------------------------
%<*def1>
%------------------------------------------------------------------------------
\bi
\item Kernel : noyau d'un système d'exploitation. Gère les ressources
      matérielles de l'ordinateur (CPU, mémoire, périphérique...). Il permet
      aux différents composant (matériel et logiciel) de communiquer entre
      eux. Il existe plusieurs architecture de kernel : monolithique,
\item Système d'exploitation : kernel plus les logiciels type compilateur,
      shell (interface utilisateur d'un OS), debuger, éditeur, ... Fournit
      une couche d'abstraction par rapport au matériel ainsi qu'une interface
      générique de programmation pour l'utilisateur / développeur.
\ei
Dans notre cas, GNU/Linux est un système d'exploitation (GNU les logiciels
end-user comme gcc, emacs, bash et Linux le kernel) : on parle donc bien de
GNU/Linux. Dire que Linux est un système d'exploitation est un abus de
langage. \\
Inclure figure6 de la doc OS embarqués.
%</def1>

%------------------------------------------------------------------------------
%<*def2>
%------------------------------------------------------------------------------
\bi
\item Système embarqué : composition d'une partie hardware (électronique)
      et d'une partie logicielle. La partie électronique est souvent très
      limitée d'un point de vue ressource (processeur, mémoire, ...) :
      Un système embarqué est généralement autonome et doit respecter un
      certain nombre de contraintes d'environnement (vibration, chaleur, ...)
      mais aussi de poids, taille, ... Exemple : Olinuxino et Raspberyy Pi. \\
      La partie HX et SW doivent prendre en compte la consommation d'énergie. \\
      La durée de vie est souvent très longue (20 ans et plus)!! Doit être
      fiable car autonome.
\ei
%</def2>

%------------------------------------------------------------------------------
%<*def3>
%------------------------------------------------------------------------------
\bi
\item Système d'exploitation embarqué : OS sur lequel un logiciel
      embarqué va être exécuté. Contrainte forte par rapport à la
      consommation en ressource matériel (comme l'empreinte mémoire, la
      consommation électrique, ...). Un OS classique n'est pas envisageable
      sur du matériel avec seulement quelques dizaines de Ko de mémoire.
\item Système d'exploitation temps réel (vs temps partagé) : garantir les
      temps de réponse (temps réel dur ou mou, préemptivité du kernel,
      latences, ... : voir le cours associé)
\ei
%</def3>

%------------------------------------------------------------------------------
%<*def4>
%------------------------------------------------------------------------------
\bi
\item Logiciel embarqué : intégration du logiciel dans un équipement
      industriel pour une application dédiée. Un bon logiciel embarqué est
      un logiciel dont on a oublié la présence. On achète un lave vaisselle
      pour qu'il fasse son travail, pas pour le logiciel qu'il y a à
      l'intérieur.
\item Linux embarqué : Kernel Linux + composants open-source pour former un OS
      embarqué sur mesure par rapport aux besoins. 8 MB de RAM minimum (bienque
      32 MB soit plus réaliste pour des applications réelles), 4 MB d'espace de
      stockage)
\ei
%</def4>

%------------------------------------------------------------------------------
%<*def5>
%------------------------------------------------------------------------------
\bi
\item Microprocesseur : séquenceur (synchronise, gère les interruptions), unité
      arithmétique et logique (UAL : calcul élémentaire), registre (mémoire
      suffisamment petite pour que l'UAL puisse manipuler leur contenu à chaque
      cycle d'horloge. Plusieurs types de registres : registre d'état pour le
      contexte en cours du processeur, pointeur de pile, accumulateur, ...),
      unité d'entrée/sortie (analogique).
\item Microcontrolleur : microprocesseur, mémoire, entrées/sorties
      analogiques (de 0 à 1)/numériques (0 ou 1)
\ei
Mais microprocesseur bcp plus puissant. Par exemple:
\bi
\item fréquence d'horloge (nombres d'instructions)
\item largeur des registres sur les nombres entiers (4 ou 8 bits contre 32 ou
      64 bits)
\ei
Remarque : registre n bits => 2\^n est le nombre d'adresses mémoire que le
           registre peut gérer. 32 bits => 4 294 967 296 de valeur non signées.
           Ainsi un processeur 32bits ne peut pas gérer plus de 4GiB de RAM
           (4x10\^9), sinon pas assez de bit dans un mot pour accéder à l'adresse
           mémoire (sauf mécanique Physical Adress Extension).
%</def5>

%------------------------------------------------------------------------------
%<*chiffres1>
%------------------------------------------------------------------------------
Voir étude OPIIEC dans mes docs slide 5.
%</chiffres1>

%------------------------------------------------------------------------------
%<*chiffres2>
%------------------------------------------------------------------------------
Voir étude OPIIEC dans mes docs slide 14.
%</chiffres2>

%------------------------------------------------------------------------------
%<*chiffres3>
%------------------------------------------------------------------------------
Voir étude OPIIEC dans mes docs slide 17.
%</chiffres3>

%------------------------------------------------------------------------------
%<*chiffres3>
%------------------------------------------------------------------------------
Voir étude OPIIEC dans mes docs slide 26.
%</chiffres3>

%------------------------------------------------------------------------------
%<*embos1>
%------------------------------------------------------------------------------
\bi
\item VxWorks : noyau temps réel conforme POSIX le plus utilisé dans
      l'industrie. Par exemple radar HF Wera déployé sur la côte bretonne!
      Licence très coûteuse...
\item QNX : noyau temps réel conforme POSIX. Gratuit pour les applications
      non commerciales.
\item micro-C OS : temps réel et de petite taille pour micro contrôleur.
      Gratuit pour l'enseignement.
\item Windows : Windows CE, Windows XP embedded, Windows Mobile, Windows
      Phone (se veut concurrent d'Android, iOS, ..., contrat avec Nokia,
      2\% du marché).
\item Plein d'autres : LynxOS, Nucleus, eCos, ...
\ei
%</embos1>

%------------------------------------------------------------------------------
%<*embos2>
%------------------------------------------------------------------------------
\bi
\item Wind River Linux : avec extension temps réel RTLinux
\item ELDK : par DENX (U-Boot), licence GPL. Fournit une distribution
      complète pour les architectures : PowerPC, ARM et MIPS. Exemple :
      boîtier Ballard!
\item Android (Google) : application en Java (quoique qu'un SDK C/C++)
\item Tizen : dernier né (2012), Open Source. EFL, Samsung : Samsung Gear
      S2. Autre gourou : Rasterman!
\item Plein d'autres : MontaVista Linux, BlueCat Linux, ...
\ei
%</embos2>

%------------------------------------------------------------------------------
%<*choice1>
%------------------------------------------------------------------------------
Poser la question : comment choisir parmi ceux ci?
\bi
\item Pas d'effet de masse donc cher.
\item Outil de développement associé peu courant donc personnel qualifié
      rare et cher.
\item Risqué... Que se passe t il si l'entreprise fournissant l'OS
      disparaît? Pas accès au code source... Or la durée de vie un système
      embarqué est relativement longue (systèmes militaire : plusieurs
      dizaines d'années! : exemple radar OM100)
\item Mais en théorie, un bon support!
\ei
%</choice1>

%------------------------------------------------------------------------------
%<*choice2>
%------------------------------------------------------------------------------
\bi
\item Redistribution sans royalties (pas de coût de licences)
\item Code source ouvert et donc modifiable (problème de licence peut être
      posé. Certaines entreprise préfèrent les licences BSD pour modifier
      le code et éviter les problèmes de redistribution!)
\item Souvent de meilleur qualité que du logiciel propriétaire!
\item une communauté contrebalançant le "as is" (exemple). Car
      pas de responsabilité si ça ne marche pas et les développeurs ne
      sont pas obligés de répondre. Peut poser problème pour les
      industries. Il faut toujours un bon coupable en cas de problème!
\item De forte chance pour avoir les composants logiciel open source pour les
      dernières avancées hard!
\ei
Dans les entreprises, bcp de Red Hat pour le support!
%</choice2>

%------------------------------------------------------------------------------
%<*linux1>
%------------------------------------------------------------------------------
\bi
\item fiable : uptime-project!
\ei
%</linux1>

%------------------------------------------------------------------------------
%<*linux2>
%------------------------------------------------------------------------------
\bi
\item pas cher : pas de royalties, outils libre
\item portabilité : x86, arm, ppc, amd, sparc, ...
\item open source : déjà vu
\ei
Remarque : on parle de la portabilité du kernel! Une distribution même avec
un kernel Linux peut apporter des contraintes par rapport à l'architecture
du CPU. Par exemple, la distribution CentOS ne peut tourner que sur des
architectures x86-64 ou i386. En revanche, une Debian supporte 11
architectures.
%</linux2>

%------------------------------------------------------------------------------
%<*linux3>
%------------------------------------------------------------------------------
\bi
\item Se retrouver dans les licences. Une seule licence propriétaire, c'est
      plus simple!
\item Beaucoup de solution pour faire la même chose (KDE, GNOME, GTK+, ...).
      Peut impliquer un sentiment de cafouillage pour les non initiés qui
      veulent un OS standard, simple, unique. Voir The cathedral And the
      Bazar.
\ei
%</linux3>

%------------------------------------------------------------------------------
%<*hard1>
%------------------------------------------------------------------------------
Kernel tourne sur de très nombreuse architectures 32 et 64 bits. Exemple de
systèmes embarqués avec Linux (Ballard, Samsung Gear, OlinuXino, Ecrin Onyx,
Getac).
Supporte les formats de mémoire flash, les bus de communications (I2C, SPI,
USB, ...). Bon support réseau : ethernet, wifi, bluetooth, IPv4, IPv6, TCP/UDP,
multicast, ...
%</hard1>

%------------------------------------------------------------------------------
%<*hard2>
%------------------------------------------------------------------------------
Memory Management Unit : unité de gestion mémoire
\bi
\item composant matériel permettant à un processus de ne jamais écraser l'espace
      mémoire d'un autre processus (segmentation fault)
\item traduction entre les addresses mémoire physique vues par le CPU et les
      addresses mémoire virtuelle vues par l'application (et allouées par l'OS)
\item composant dédié à l'époque mais maintenant intégré aux microprocesseurs
\ei

Depuis la version 2.5.46 du kernel Linux, les processeurs sans MMU (dit MMU-less)
sont supportés via la ucLibc (glibc ou eglibc nécessite une MMU!). Vient à la
base du fork du Kernel ucLinux. Actuellement, le faible coût des processeurs
avec MMU rend cela très rare...

Au moment du choix du hard, vérifier qu'il existe un support OFICIEL du kernel
Linux. Sinon temps/coût de développement bcp plus élevé!
%</hard2>
