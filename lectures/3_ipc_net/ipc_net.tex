\documentclass[12pt,t]{beamer}

%------------------------------------------------------------------------------
% configuration
%------------------------------------------------------------------------------
\RequirePackage{etex}
\usepackage{../../themes/dbt}
\usepackage{catchfilebetweentags}

\setbeameroption{hide notes}
\setbeamertemplate{caption}{\raggedright\insertcaption\par}

\graphicspath{{images/}}

% a few macros
\newcommand{\bi}{\begin{itemize}}
\newcommand{\ei}{\end{itemize}}
\newcommand{\ig}{\includegraphics}
\newcommand{\myhref}[1]{\href{#1}{\tt \scriptsize #1}}
\newcommand{\incnote}[1]{\note{\ExecuteMetaData[notes.tex]{#1}}}
\newcommand{\src}[2]{\vspace{-10pt}\caption{\href{#1}{\centering \tt \tiny [#2]}}}

%------------------------------------------------------------------------------
% title
%------------------------------------------------------------------------------
% slide
\title{Systèmes d'exploitation pour l'embarqué}
\subtitle{UV 5.2 - Exécution et Concurrence}

\author{\href{}{Paul Blottière}}
\institute{
    \href{http://www.ensta-bretagne.fr/}{ENSTA Bretagne} \\[2pt]
    \href{}{\tt \scriptsize 24 Novembre 2015}
}
\date{
    \href{https://github.com/pblottiere}{\tt \scriptsize https://github.com/pblottiere} \\[2pt]
    %\href{blottiere.paul@gmail.com}{\tt \scriptsize blottiere.paul@gmail.com}
}

% info
\begin{document}

{
\setbeamertemplate{footline}{} % no page number here
\frame{
    \titlepage
    \incnote{title}
} }

%------------------------------------------------------------------------------
% amélioration continue
%------------------------------------------------------------------------------
\begin{frame}{Amélioration continue}
    \subt{Contributions}
    \vspace{12pt}

    \begin{center}
    \includegraphics[scale=0.7]{github.png}
    \end{center}

    \bi
    \itemsep12pt
    \item Dépôt du cours : \href{https://github.com/pblottiere/embsys}{\tt \scriptsize https://github.com/pblottiere/embsys}

    \onslide<2->{
        \item Souhaits d'amélioration, erreurs, idées de TP, ... : ouverture d'Issues (avec le bon label!)
        \item Apports de corrections : Pull Request
        \ei
    }
\end{frame}

%<**lecture_content>
%------------------------------------------------------------------------------
% lecture
%------------------------------------------------------------------------------
\begin{frame}[plain,c]
    \centering
    \huge\textcolor{title}{IPC et programmation réseau}
\end{frame}

%------------------------------------------------------------------------------
% plan
%------------------------------------------------------------------------------
\begin{frame}{Plan}
    \subt{}
    \vspace{10pt}

    \begin{enumerate}
        \itemsep12pt
        \item Définitions
        \item Outils
        \item Appels système
        \item Identification (PID, ...)
        \item Capacités d'un processus
        \item Création de processus
        \item Terminaison d'un processus
        \item Les pthreads
    \end{enumerate}

    \note {
    }
\end{frame}

%------------------------------------------------------------------------------
% conclusion
%------------------------------------------------------------------------------
\begin{frame}{Conclusion}

    \centering
    \vspace{20pt}
    \LARGE{
    }

\end{frame}

%------------------------------------------------------------------------------
% ref
%------------------------------------------------------------------------------
\begin{frame}{Références}
    \vspace{30pt}

    \bi
    \itemsep12pt
    \item Linux Embarqué - Pierre Ficheux
    \item Développement système sous Linux - Christophe Blaess
    \item Modern Operating Systems - Andrew Tanenbaum
    \ei
\end{frame}
%<//lecture_content>

\end{document}
