\documentclass[12pt,t]{beamer}

%------------------------------------------------------------------------------
% configuration
%------------------------------------------------------------------------------
\RequirePackage{etex}
\usepackage{currfile-abspath}
\usepackage{../../themes/dbt}
\usepackage{catchfilebetweentags}

\setbeameroption{hide notes}
\setbeamertemplate{caption}{\raggedright\insertcaption\par}

\graphicspath{{images/}}
\getmainfile
\getabspath{\themainfile}
\let\mainabsdir\theabsdir
\let\mainabspath\theabspath

\newcommand{\insertcode}[2]{\lstinputlisting[label=samplecode, basicstyle=#1]{\mainabsdir/code/#2}}
\newcommand{\bi}{\begin{itemize}}
\newcommand{\ei}{\end{itemize}}
\newcommand{\ig}{\includegraphics}
\newcommand{\myhref}[1]{\href{#1}{\tt \scriptsize #1}}
\newcommand{\incnote}[1]{\note{\ExecuteMetaData[notes.tex]{#1}}}
\newcommand{\src}[2]{\vspace{-10pt}\caption{\href{#1}{\centering \tt \tiny [#2]}}}


%------------------------------------------------------------------------------
% title
%------------------------------------------------------------------------------
%------------------------------------------------------------------------------
% title
%------------------------------------------------------------------------------
% slide
\title{Systèmes d'exploitation pour l'embarqué}
\subtitle{UV 5.2 - Exécution et Concurrence}

\author{\href{}{Paul Blottière}}
\institute{
    \href{http://www.ensta-bretagne.fr/}{ENSTA Bretagne} \\[2pt]
    \href{}{\tt \scriptsize 2018 / 2019}
}
\date{
    \href{https://github.com/pblottiere}{\tt \scriptsize https://github.com/pblottiere} \\[2pt]
    %\href{blottiere.paul@gmail.com}{\tt \scriptsize blottiere.paul@gmail.com}
}

% info
\begin{document}

{
\setbeamertemplate{footline}{} % no page number here
\frame{
    \titlepage
} }

%------------------------------------------------------------------------------
% amélioration continue
%------------------------------------------------------------------------------
\begin{frame}{Amélioration continue}
    \subt{Contributions}
    \vspace{12pt}

    \begin{center}
    \includegraphics[scale=0.7]{github.png}
    \end{center}

    \bi
    \itemsep12pt
    \item Dépôt du cours : \href{https://github.com/pblottiere/embsys}{\tt \scriptsize https://github.com/pblottiere/embsys}
    \item Souhaits d'amélioration, erreurs, idées de TP, ... : ouverture d'Issues
    \item Apports de corrections : Pull Request
    \ei
\end{frame}




%------------------------------------------------------------------------------
% Questions de cours
%------------------------------------------------------------------------------
\begin{frame}{Cours}
    \subt{Test de connaissance}

    \vspace{10pt}
    \bi
    \itemsep12pt
    \item 1 test de connaissance à la fin du semestre
    \item 3 questions par cours
    \item Questions de vocabulaires, concepts, terminologie, ...
    \item Bachotage :)
    \ei

\end{frame}

%------------------------------------------------------------------------------
% TP
%------------------------------------------------------------------------------
\begin{frame}{TP}
    \subt{Comptes rendu}

    \vspace{10pt}
    \bi
    \itemsep12pt
    \item Par groupe de 2 ou 3 max (garder les groupes pour le projet)
    \item Plusieurs comptes-rendu au cours du semestre
    \item À rendre d'une séance sur l'autre
    \item Tarball (.tar.gz)
    \item Notes en Markdown
    \item Code source avec Makefile (-Wall)
    \ei

\end{frame}

%------------------------------------------------------------------------------
% projet
%------------------------------------------------------------------------------
\begin{frame}{Projet}
    \subt{Volume horaire : 2 demi-journées}

    \begin{enumerate}
        \itemsep3pt
        \item Fonctionnel
        \item Workflow git et collaboration (branche, tag, PR, ...)
        \item Architecture du code et modularité (librairies, ...)
        \item Outils de build (autotools)
        \item Utilisation des notions vues en TP (getopt, syslog, signal handler, ...)
        \item Normes de codage
        \item Compilation avec -Wall
        \item Tests unitaires
        \item Readme global (markdown)
        \item Documentation sur l'utilisation des outils (docker / buildroot / QEMU)
        \item Documentation du code (doxygen)
    \end{enumerate}

\end{frame}

\end{document}
